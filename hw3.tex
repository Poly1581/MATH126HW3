\documentclass{article}

\usepackage[usenames,dvipsnames]{xcolor}
\usepackage{fullpage,latexsym,picinpar,amsmath,amsfonts,graphicx, tikz, subcaption, upgreek}
\usetikzlibrary{trees, decorations, shadows}

%%%%%%%%%%%%%%%%%%%%%%%%%%%%%%%%%%%%%%%%%%%%%%%%%%%%%%%%%%%%%%%%%%%%%%%%%%%%%%%%%%%
%%%%%%%%%%%  LETTERS 
%%%%%%%%%%%%%%%%%%%%%%%%%%%%%%%%%%%%%%%%%%%%%%%%%%%%%%%%%%%%%%%%%%%%%%%%%%%%%%%%%%%

\newcommand{\barx}{{\bar x}}
\newcommand{\bary}{{\bar y}}
\newcommand{\barz}{{\bar z}}
\newcommand{\bart}{{\bar t}}

\newcommand{\bfP}{{\bf{P}}}

%%%%%%%%%%%%%%%%%%%%%%%%%%%%%%%%%%%%%%%%%%%%%%%%%%%%%%%%%%%%%%%%%%%%%%%%%%%%%%%%%%%
%%%%%%%%%%%%%%%%%%%%%%%%%%%%%%%%%%%%%%%%%%%%%%%%%%%%%%%%%%%%%%%%%%%%%%%%%%%%%%%%%%%
                                                                                
\newcommand{\parend}[1]{{\left( #1  \right) }}
\newcommand{\spparend}[1]{{\left(\, #1  \,\right) }}
\newcommand{\angled}[1]{{\left\langle #1  \right\rangle }}
\newcommand{\brackd}[1]{{\left[ #1  \right] }}
\newcommand{\spbrackd}[1]{{\left[\, #1  \,\right] }}
\newcommand{\braced}[1]{{\left\{ #1  \right\} }}
\newcommand{\leftbraced}[1]{{\left\{ #1  \right. }}
\newcommand{\floor}[1]{{\left\lfloor #1\right\rfloor}}
\newcommand{\ceiling}[1]{{\left\lceil #1\right\rceil}}
\newcommand{\barred}[1]{{\left|#1\right|}}
\newcommand{\doublebarred}[1]{{\left|\left|#1\right|\right|}}
\newcommand{\spaced}[1]{{\, #1\, }}
\newcommand{\suchthat}{{\spaced{|}}}
\newcommand{\numof}{{\sharp}}
\newcommand{\assign}{{\,\leftarrow\,}}
\newcommand{\myaccept}{{\mbox{\tiny accept}}}
\newcommand{\myreject}{{\mbox{\tiny reject}}}
\newcommand{\blanksymbol}{{\sqcup}}
                                                                                                                         
\newcommand{\veps}{{\varepsilon}}
\newcommand{\Sigmastar}{{\Sigma^\ast}}
                           
\newcommand{\half}{\mbox{$\frac{1}{2}$}}    
\newcommand{\threehalfs}{\mbox{$\frac{3}{2}$}}   
\newcommand{\domino}[2]{\left[\frac{#1}{#2}\right]}  

%%%%%%%%%%%% complexity classes

\newcommand{\PP}{\mathbb{P}}
\newcommand{\NP}{\mathbb{NP}}
\newcommand{\PSPACE}{\mathbb{PSPACE}}
\newcommand{\coNP}{\textrm{co}\mathbb{NP}}
\newcommand{\DLOG}{\mathbb{L}}
\newcommand{\NLOG}{\mathbb{NL}}
\newcommand{\NL}{\mathbb{NL}}

%%%%%%%%%%% decision problems

\newcommand{\PCP}{\sc{PCP}}
\newcommand{\Path}{\sc{Path}}
\newcommand{\GenGeo}{\sc{Generalized Geography}}

\newcommand{\malytm}{{\mbox{\tiny TM}}}
\newcommand{\malycfg}{{\mbox{\tiny CFG}}}
\newcommand{\Atm}{\mbox{\rm A}_\malytm}
\newcommand{\complAtm}{{\overline{\mbox{\rm A}}}_\malytm}
\newcommand{\AllCFG}{{\mbox{\sc All}}_\malycfg}
\newcommand{\complAllCFG}{{\overline{\mbox{\sc All}}}_\malycfg}
\newcommand{\complL}{{\bar L}}
\newcommand{\TQBF}{\mbox{\sc TQBF}}
\newcommand{\SAT}{\mbox{\sc SAT}}

%%%%%%%%%%%%%%%%%%%%%%%%%%%%%%%%%%%%%%%%%%%%%%%%%%%%%%%%%%%%%%%%%%%%%%%%%%%%%%%%%%%
%%%%%%%%%%%%%%% for homeworks
%%%%%%%%%%%%%%%%%%%%%%%%%%%%%%%%%%%%%%%%%%%%%%%%%%%%%%%%%%%%%%%%%%%%%%%%%%%%%%%%%%%

\newcommand{\student}[2]{%
{\noindent\Large{ \emph{#1} SID {#2} } \hfill} \vskip 0.1in}

\newcommand{\assignment}[1]{\medskip\centerline{\large\bf CS 111 ASSIGNMENT {#1}}}

\newcommand{\duedate}[1]{{\centerline{due {#1}\medskip}}}     

\newcounter{problemnumber}                                                                                 

\newenvironment{problem}{{\vskip 0.1in \noindent
              \bf Problem~\addtocounter{problemnumber}{1}\arabic{problemnumber}:}}{}

\newcounter{solutionnumber}

\newenvironment{solution}{{\vskip 0.1in \noindent
             \bf Solution~\addtocounter{solutionnumber}{1}\arabic{solutionnumber}:}}
				{\ \newline\smallskip\lineacross\smallskip}

\newcommand{\lineacross}{\noindent\mbox{}\hrulefill\mbox{}}

\newcommand{\decproblem}[3]{%
\medskip
\noindent
\begin{list}{\hfill}{\setlength{\labelsep}{0in}
                       \setlength{\topsep}{0in}
                       \setlength{\partopsep}{0in}
                       \setlength{\leftmargin}{0in}
                       \setlength{\listparindent}{0in}
                       \setlength{\labelwidth}{0.5in}
                       \setlength{\itemindent}{0in}
                       \setlength{\itemsep}{0in}
                     }
\item{{{\sc{#1}}:}}
                \begin{list}{\hfill}{\setlength{\labelsep}{0.1in}
                       \setlength{\topsep}{0in}
                       \setlength{\partopsep}{0in}
                       \setlength{\leftmargin}{0.5in}
                       \setlength{\labelwidth}{0.5in}
                       \setlength{\listparindent}{0in}
                       \setlength{\itemindent}{0in}
                       \setlength{\itemsep}{0in}
                       }
                \item{{\em Instance:\ }}{#2}
                \item{{\em Query:\ }}{#3}
                \end{list}
\end{list}
\medskip
}

%%%%%%%%%%%%%%%%%%%%%%%%%%%%%%%%%%%%%%%%%%%%%%%%%%%%%%%%%%%%%%%%%%%%%%%%%%%%%%%%%%%
%%%%%%%%%%%%% for quizzes
%%%%%%%%%%%%%%%%%%%%%%%%%%%%%%%%%%%%%%%%%%%%%%%%%%%%%%%%%%%%%%%%%%%%%%%%%%%%%%%%%%%

\newcommand{\quizheader}{ {\large NAME: \hskip 3in SID:\hfill}
                                \newline\lineacross \medskip }

%\newcommand{\namespace}{ {\large NAME: \hskip 3in SID:\hfill}
%                               \newline\lineacross \medskip }

%%%%%%%%%%%%%%%%%%%%%%%%%%%%%%%%%%%%%%%%%%%%%%%%%%%%%%%%%%%%%%%%%%%%%%%%%%%%%%%%%%%
%%%%%%%%%%%%% for final
%%%%%%%%%%%%%%%%%%%%%%%%%%%%%%%%%%%%%%%%%%%%%%%%%%%%%%%%%%%%%%%%%%%%%%%%%%%%%%%%%%%

\newcommand{\namespace}{\noindent{\Large NAME: \hfill SID:\hskip 1.5in\ }\\\medskip\noindent\mbox{}\hrulefill\mbox{}}



\begin{document}

\centerline{\large \bf MATH126 Homework 3}

\vskip 0.15in

\tikzstyle{plain} = [shape = circle, draw = black]
\tikzstyle{dir} = [->, line width=0.5mm]
%-----------------------------------Finished-------------------------------------
%-----------------------------------Problem 1------------------------------------
\begin{problem} The complete bipartite graph $K_{l,r}$ is formed by taking one set of $l$ vertices (which I will call the left set or $L$ and one set of $r$ vertices (which I will call the right set or $R$), and connecting every vertex on one set to every vertex in the other set.  For example, $K_{2,3}$ is shown below:
\vskip 0.15in \noindent
\begin{tikzpicture}
	\node[plain] at (0,0) (lu) {};
	\node[plain] at (0,1) (ld) {};
	\node[plain] at (1,1.5) (ru) {};
	\node[plain] at (1,0.5) (rm) {};
	\node[plain] at (1,-0.5) (rd) {};
	\path (lu) edge (ru) edge (rm) edge (rd);
	\path (ld) edge (ru) edge (rm) edge (rd);
\end{tikzpicture}
\vskip 0.15in \noindent
Under what conditions on l and m does $K_{l,r}$ have a Hamiltonian cycle? (In particular, you should explain why the other pairs $(l,r)$ do \emph{not} have a cycle. It's not enough to give a few examples - you should (at least briefly - explain why it's true in general!)
\end{problem}

\begin{solution} A Hamiltonian cycle is a circuit (a path that starts and ends at the same vertex) that passes through (or includes) all of the vertices of the graph.  From this definition, it follows that all vertices in the graph must be connected through 2 edges in the cycle (one in and one out, though in and out are arbitrary in an undirected graph).  Since the graphs we are considering are bipartite graphs, all edges connect one vertex in the left group and one vertex in the right group.  Combining all of this, the left group must have a total of $2\cdot l$ connections to vertices in the right group, and the right group must have $2\cdot r$ connections to vertices in the left group, so $2\cdot l = 2\cdot r \Rightarrow r=l$.  The left and right groups in a complete bipartite graph must be equal in size for the graph to have a Hamiltonian cycle.
\end{solution}


%-----------------------------------Finished-------------------------------------
%-----------------------------------Problem 2------------------------------------
\begin{problem} In class, we showed that every graph on $n$ vertices where each vertex has degree at least $n/2$ automatically has a Hamiltonian cycle.
\begin{itemize}
	\item[]{\textbf{Part a:}} Show that for any $m$, there is a graph on $2m+1$ vertices where each vertex has degree at least $m$ that does \emph{not} have a Hamiltonian cycle.
	\item[]{\textbf{Part b:}} Give an example of a graph on 6 vertices with 11 edges that does not have a Hamiltonian Cycle.
\end{itemize}
\end{problem}

\begin{solution}
\begin{itemize}
	\item[]{\textbf{Part a:}} I will use the solution of problem 1 to solve this.  We can construct a complete bipartite graph on $2m+1$ vertices where, without loss of generality, the left side has $m$ vertices and the right side has $m+1$ vertices.  In this graph, every vertex in the left side would have $m+1$ edges, and every vertex on the right would have $m$ edges - so every vertex has degree at least $m$.  From problem 1, we have shown that a complete bipartite graph with the number of vertices in each side being nonequal necessarily means that such a graph cannot have a Hamiltonian cycle, so this graph does not have a Hamiltonian cycle.
	\item[]{\textbf{Part b:}} Such a graph - with $6$ vertices and $11$ edges can be constructed as follows:
	\newline\noindent
	\begin{tikzpicture}
		\node[plain] at (2,0) (a) {};
		\node[plain] at (0.61803398875,1.90211303259) (b) {};
		\node[plain] at (-1.61803398875,1.17557050458) (c) {};
		\node[plain] at (-1.61803398875,-1.17557050458) (d) {};
		\node[plain] at (0.61803398875,-1.90211303259) (e) {};
		\node[plain] at (4,0) (f) {};
		\path (a) edge (b) edge (c) edge (d) edge (e) edge (f);
		\path (b) edge (c) edge (d) edge (e);
		\path (c) edge (d) edge (e);
		\path (d) edge (e);
	\end{tikzpicture}
	\vskip 0.15in\noindent
	It can be seen that this graph has $6$ vertices and $11$ edges, but cannot possibly have a Hamiltonian cycle since the rightmost vertex has degree 1, meaning that it is impossible for any cycle to include it.
\end{itemize}
\end{solution}


%-----------------------------------Finished-------------------------------------
%-----------------------------------Problem 3------------------------------------
\begin{problem} Consider the De Bruijn graph (as defined in lecture/lecture notes) with $L = \{0,1\}$ and $k=4$.
\begin{itemize}
	\item[]{\textbf{Part a:}} How many vertices does this graph have? How many (directed) edges does it have?
	\item[]{\textbf{Part b:}} Draw the graph.
	\item[]{\textbf{Part c:}} Find an Eulerian Circuit on the graph. What is the corresponding De Bruijn sequence?
\end{itemize}
\end{problem}

\begin{solution}
\begin{itemize}
	\item[]{\textbf{Part a:}} Every vertex corresponds to a unique binary sequence of length 3, so there are $2^3=8$ vertices.
	\item[]{\textbf{Part b:}} The De Bruijn grpah with $L=\{0,1\}$ and $k=4$ is as follows:
	\vskip 0.15in \noindent
	\begin{tikzpicture}
		\node[plain] at (10,0) (000) {000};
		\node[plain] at (8,-2) (001) {001};
		\node[plain] at (6,0) (010) {010};
		\node[plain] at (2,-2) (011) {011};
		\node[plain] at (8,2) (100) {100};
		\node[plain] at (4,0) (101) {101};
		\node[plain] at (2,2) (110) {110};
		\node[plain] at (0,0) (111) {111};
		\draw[dir] (000) to [loop right, out=320, in=45, distance=10mm] node[midway, right] {+0} (000);
		\draw[dir] (000) to node[midway, above left, sloped, pos=0.3] {+1} (001);
		\draw[dir] (001) to node[midway, above right, sloped, pos=0.65] {+0} (010);
		\draw[dir] (001) to node[midway, above] {+1} (011);
		\draw[dir] (010) to node[midway, above left, sloped, pos=0.65] {+0} (100);
		\draw[dir] (010) to [out=135, in=45] node[midway, above] {+1} (101);		
		\draw[dir] (011) to node[midway, left] {+0} (110);
		\draw[dir] (011) to node[midway, above left, sloped, pos=0.3] {+1} (111);
		\draw[dir] (100) to node[midway, above right, sloped, pos=0.3] {+0} (000);
		\draw[dir] (100) to node[midway, right] {+1} (001);
		\draw[dir] (101) to [out=320, in=225] node[midway, below] {+0} (010);
		\draw[dir] (101) to node[midway, above left, sloped, pos=0.3] {+1} (011);
		\draw[dir] (110) to node[midway, above] {+0} (100);
		\draw[dir] (110) to node[midway, above right, sloped, pos=0.3] {+1} (101);
		\draw[dir] (111) to node[midway, above left, sloped, pos=0.65] {+0} (110);
		\draw[dir] (111) to [loop left, out=225, in=135, distance=10mm] node[midway, left] {+1} (111);
	\end{tikzpicture}
	\item[]{\textbf{Part c:}} An Eulerian circuit in the De Brujin graph form \textbf{part b} is as follows : 000,001,011,111,110,101,010,100,000, and the corresponding De Brujin sequence is 11101000.
\end{itemize}
\end{solution}


%-----------------------------------Finished-------------------------------------
%-----------------------------------Problem 4------------------------------------
\begin{problem} In class, we showed that it is impossible to draw the complete graph $K_5$ without having at least one pair of crossing edges. Show that it is possible to draw $K_5$ such that there is only one pair of edges that cross.
\end{problem}

\begin{solution}
One way to show that something is possible is to actually do it - in this case that means that I will draw the graph $K_5$ with only one crossing edge.  Typically,$K_5$ is drawn as follows:
\vskip 0.15in \noindent
\begin{tikzpicture}
	\node[plain] at (2,0) (a) {};
	\node[plain] at (0.61803398875,1.90211303259) (b) {};
	\node[plain] at (-1.61803398875,1.17557050458) (c) {};
	\node[plain] at (-1.61803398875,-1.17557050458) (d) {};
	\node[plain] at (0.61803398875,-1.90211303259) (e) {};
	\path (a) edge (b) edge (c) edge (d) edge (e);
	\path (b) edge (c) edge (d) edge (e);
	\path (c) edge (d) edge (e);
	\path (d) edge (e);
\end{tikzpicture}
\vskip 0.15in \noindent
It doesn't take a mathematician to see that this graph has many edges that cross many times (a total of 5 edge crossings). We can redraw the graph so that it only has one pair of crossing edges as follows:
\vskip 0.15in \noindent
\begin{tikzpicture}
	\node[plain] at (0,0) (top) {};
	\node[plain] at (-0.75,-2.5) (middle left) {};
	\node[plain] at (0.75,-2.5) (middle right) {};
	\node[plain] at (-2,-4) (bottom left) {};
	\node[plain] at (2,-4) (bottom right) {};
	\path (top) edge (middle left) edge (middle right) edge (bottom left) edge (bottom right);
	\path (middle left) edge (bottom left) edge (middle right) edge (bottom right);
	\path (middle right) edge (bottom left) edge (bottom right);
	\path (bottom left) edge (bottom right);
\end{tikzpicture}
\vskip 0.15in \noindent
This graph can clearly be seen to be both equivalent to the previous drawing of $K_5$ and to have only one pair of edges that cross, so we have shown that it is possible to draw $K_5$ with only one pair of edges that overlap.
\end{solution}


%-----------------------------------Finished-------------------------------------
%-----------------------------------Problem 5------------------------------------
\begin{problem} In class on Wednesday, we showed that if $G$ has $v$ vertices, $e$ edges, and a planar drawing with $r$ regions, then $v-e+r=2$.
\begin{itemize}
	\item[]{\textbf{Part a:}} Show that the assumption of "Connected" is necessary in the above Theorem by giving an example of a planar graph with $v-e+r \neq 2$.
	\item[]{\textbf{Part b:}} Suppose that G has $v$ vertices, $e$ edges, $k$ components, and a planar drawing with $r$ regions.  Determine an equation relating $v$, $e$, $r$, and $k$. (You may want to look at the Proof of Euler's Theorem from lecture and determine what needs to be changed).
\end{itemize}
\end{problem}

\begin{solution}
\begin{itemize}
	\item[]{\textbf{Part a:}} Consider the following unconnected graph:
		\vskip 0.15in \noindent
		\begin{tikzpicture}
			\node[plain] (a) {};
			\node[plain] (b) [right of=a]{}; 
		\end{tikzpicture}
		\vskip 0.15in \noindent
		Where $v = 2$, $e = 0$, $r = 1$, and $v - e + r = 3$.  It is clear that this graph, 		because it is unconnected, violates Euler's formula for planar graphs.
	\item[]{\textbf{Part b:}} Consider the set of connected components of an unconnected graph.  It is clear that if an unconnected graph admits a planar drawing (i.e. it is planar), each of its connected components must themselves be planar.  This means that Euler's formula for planar graphs applies to each of the components. If we write that for each component $k$, $v_k$ is the number of vertices in that component, $e_k$ is the number of edges of that component, and $r_k$ is the number of regions for that component, we can then write $\forall k, \quad v_k-e_k+r_k=2$.  Since each component is not connected to any component, no vertices nor edges in any component are included in another - such that $\sum_k v_k = v$, and $\sum_k e_k = e$.  When we consider how to combine the regions, we need to be a bit more careful.  Every component has both internal regions and an external region - the region entierely unbounded by the component.  Since the external region is not bounded by any of the components, all components share the same external region, so we can write $\sum_k r_k = r + k - 1$. Now we have all that we need to complete the analog to Euler's formula for unconnected graphs: $\sum_k (v_k-e_k+r_k) = \sum_k v_k - \sum_k e_k + \sum_k r_k = v - e + r + k - 1$,  and $\sum_k 2 = 2 \cdot k$, so we obtain the equation $v-e+r+k-1=2\cdot k$, which after some algebraic persuasion gives the final result:
$$v-e+r-k=1$$
	
\end{itemize}
\end{solution}

\end{document}